\newif\ifpdf
\ifx\pdfoutput\undefined
\pdffalse % pdfLaTeX не используется
\else
\pdfoutput=1 % используется PDFLaTeX
\pdftrue
\fi


\documentclass[a4paper,12pt]{article}


\ifpdf
\usepackage{cmap} % чтобы работал поиск по PDF
% \usepackage[pdftex]{graphicx}
\pdfcompresslevel=9 % сжимать PDF
\else
\usepackage{graphicx}
\fi


\usepackage[colorlinks,unicode]{hyperref}

\usepackage[utf8]{inputenc}
\usepackage[english,russian]{babel}

\usepackage{booktabs}
%\usepackage{multirow}
%\usepackage{longtable}

\usepackage{epsfig}

\usepackage{amssymb}
\usepackage{amsmath}
\usepackage{wasysym}

\textheight=27.0cm
\topskip=1cm
\textwidth=18.5cm
%\oddsidemargin=-0.46cm
\voffset=-40mm
\hoffset=-25mm

\setlength{\parindent}{1cm}
\setlength{\parskip}{0.3cm}

% для того, чтобы можно было переносить от 2 и более букв в слове
\righthyphenmin=2

% нормальная обработка переносов строк, чтобы не возникало Over\Underfull \hbox.
% Хотя.. говорят, что мол ее использовать только в черновых вариантах, Но у нас же не книга в типографию. Сойдет :)
\sloppy

% устанавливаем полуторный межстрочный интервал
% \renewcommand{\baselinestretch}{1.5}

% устанавливаем формат нумерации enumerate-списков (нумерованных списков)
\renewcommand{\labelenumi}{\arabic{enumi})}
\renewcommand{\labelenumii}{\arabic{enumi}.\arabic{enumii})}

\begin{document}

created by : Кунцьо Степан aka StepLg (421) \\
version/date : 0.3 / 14.11.2008 \\
thanks to : Азимов Александр aka mitradir (421) за доскональную проверку \\
mailto : StepLg@GMail.com \\
homepage : http://github.com/StepLg/book-mathlog-theormin \\

\textbf{предикат} --- термин, обозначающий член предложения --- сказуемое.

\textbf{предикат} --- отношение между лицами, предметами, событиями, явлениями.

\textbf{логика предикатов} --- изучает законы причинно-следственной зависимости между событиями, представленными в виде отношений.

Язык логики предикатов определяется
\begin{itemize}
 \item алфавитом,
 \item синтаксисом,
 \item семантикой
\end{itemize}

Базовые символы
\begin{itemize}
 \item предметные переменные (\textbf{Var}) --- имена предметов
 \item предметные константы (\textbf{Const}) ---
 \item функциональные символы (\textbf{Func}) --- операции над предметами
 \item предикатные символы (\textbf{Pred}) --- отношения между предметами
\end{itemize}

Сигнатура алфавита - тройка < Const, Pred, Func >

Логические связки и кванторы
\begin{itemize}
 \item конъюнкция
 \item дизъюнкция
 \item отрицание
 \item импликация
 \item квантор всеобщности
 \item квантор существования
\end{itemize}

Знаки препинания: разделитель запятая и скобки

Термы --- предметы, вступающие в отношения друг с другом.

\textbf{Терм} --- рекурсивное определение (2.8)

Term --- множество термов заданного алфавита

\textbf{$Var_t$} --- множество переменных, входящих в состав заданного терма t

\textbf{Основной терм} --- такой терм t, что его $Var_t = \varnothing$

\textbf{Формула} --- рекурсивное определение (2.10) через \textbf{атомарную формулу} и \textbf{составную формулу}

Form --- множество всех формул заданного алфавита

Квантор связывает ту переменную, которая следует за ним.

\textbf{Связанное вхождение переменной} --- вхождение этой переменной в области действия квантора, связывающего ее.

\textbf{Свободное вхождение переменной} --- вхождение, не являющееся связанным.

\textbf{Свободная переменная} --- переменная, имеющая свободное вхождение в формулу.

\textbf{$Var_{\varphi}$} --- множество всех свободных переменных формулы $\varphi$.

\textbf{Замкнутая формула} (предложение) --- такая формула, у которой $Var_{\varphi} = \varnothing$.

\textbf{Приоритеты логических операций}: $\{\neg ,\exists ,\forall\}~,~\{\wedge\}~,~\{\vee\}~,~\{\rightarrow\}$

Семантика --- это свод правил, наделяющих значением (смыслом) синтаксические конструкции языка.

Интерпретации --- алгебраические системы, определяющие значения термов и формул.

Интерпретации --- это математические миры, в которых оценивается выполнимость отношений, представленных логическими формулами.

\textbf{Интерпретация сигнатуры} $<Const, Func, Pred>$ --- это алгебраическая система $I = < D_I , \overline{Const}, \overline{Func}, \overline{Pred} >$, где
\begin{itemize}
 \item $D_I$ --- непустое множество, которое называется областью интерпретации , предметной областью , или универсумом ;
 \item $\overline{Const} : Const \rightarrow D_I$ --- оценка констант , сопоставляющая каждой константе c элемент (предмет) $\overline{c}$ из области интерпретации;
 \item $\overline{Func} : Func^{(n)} \rightarrow (D_I^n \rightarrow D_I )$ --- оценка функциональных символов , сопоставляющая каждому функциональному символу $f^{(n)}$ местности n всюду определенную n-местную функцию $\overline{f}^{(n)}$ на области   интерпретации;
 \item $\overline{Pred} : Pred^{(m)} \rightarrow (D_I^m \rightarrow \{true, false\})$ --- оценка предикатных символов, сопоставляющая каждому предикатному символу $P^{(m)}$ местности m всюду определенное m-местное отношение $\overline{P}^{(m)}$ на области интерпретации.
\end{itemize}

\textbf{Значение терма} --- рекурсивно (2.25)

\textbf{Отношение выполнимости} $I \models \varphi(x_1 , x_2 , \dots , x_n )[d_1 , d_2 , \dots , d_n ]$
формулы $\varphi$ в интерпретации $I$ на наборе $d_1 , d_2 , \dots , d_n $ определяется рекурсивно (2.27-30)



%%%%%%%%%%%%%%%%%%%%%%%%%%%Лекция 3%%%%%%%%%%%%%%%%%%%%%%%%%%%%%
Формула $\varphi(x_1 , \dots, x_n )$ называется \textbf{выполнимой в интерпретации} I, если существует такой набор элементов $d_1 , \dots, d_n \in D_I$ , для которого имеет место $I \models \varphi(x_1 , \dots, x_n )[d_1 , \dots , d_n ]$.

Формула $\varphi(x_1 , \dots, x_n )$ называется \textbf{истинной в интерпретации} I, если для любого набора элементов $d_1 , \dots, d_n \in D_I$, для которого имеет место $I \models \varphi(x_1 , \dots, x_n )[d_1 , \dots , d_n ]$.

Формула $\varphi(x_1 , \dots, x_n )$ называется \textbf{выполнимой}, если есть интерпретация I, в которой эта формула выполнима.

Формула $\varphi(x_1 , \dots, x_n )$ называется \textbf{общезначимой} (или тождественно истинной ), если эта формула истинна в любой интерпретации.

Формула $\varphi(x_1 , \dots, x_n )$ называется \textbf{противоречивой} (или невыполнимой ), если она не является выполнимой.

\textbf{CForm} --- множество всех замкнутых формул.

\textbf{Модель} для множества замкнутых формул Г$ \subseteq CForm$ - любая интерпретация I, в которой выполняются все формулы множества Г.

Любая интерпретация является моделью пустого множества $\text{Г} = \varnothing$

Замкнутая формула $\varphi$ является \textbf{логическим следствием} множества замкнутых формул Г (база знаний), если каждая модель для Г является моделью для формулы $\varphi$, т.е. для любой интерпретации $I \models \text{Г} \Rightarrow I \models \varphi$. Обозначается $\text{Г} \models \varphi$

Обозначение общезначимых формул: $ \models \varphi$

\textbf{Теорема о логическом следствии}. Пусть Г$ = \{\psi_1 , \dots, \psi_n\} \subseteq CForm, \varphi \in CForm$. Тогда Г$ \models \psi ~ \Leftrightarrow ~ \models \psi_1 \wedge \dots \wedge \psi_n \rightarrow \varphi.$

Существует такая замкнутая формула $\varphi$, которая истинна в любой интерпретации I с конечной предметной областью $D_I$ , но не является общезначимой. $\forall x \neg R(x, x) \wedge \forall x \forall y \forall z(R(x,y) \wedge R(y,z) \rightarrow R(x,z)) \rightarrow \exists x \forall y \neg R(x,y)$.

\textbf{Семантическая таблица} --- это упорядоченная пара множеств формул $<\text{Г};\Delta>$, $\text{Г},\Delta \subseteq Form$. При этом Г --- это множество формул, которые мы хотим считать истинными, а $\Delta$ --- это множество формул, которые мы хотим считать ложными.

\textbf{Выполнимая семантическая таблица} $<\text{Г};\Delta>$ --- такая таблица, что существует такая интерпретация I и такой набор значений $d_1 , d_2 , \dots , d_n \in D_I$ свободных переменных $\{x_1, x_2, \dots, x_n\}$ в формулах множеств Г, $\Delta$, для которых
\begin{itemize}
 \item $I \models \varphi(x_1 , x_2 , \dots , x_n )[d_1 , d_2 , \dots , d_n ]$ для любой формулы $\varphi, \varphi \in \text{Г}$
 \item $I \nvDash \psi(x_1 , x_2 , \dots , x_n )[d_1 , d_2 , \dots , d_n ]$ для любой формулы $\psi, \psi \in \Delta$
\end{itemize}

\textbf{Теорема о табличной проверке общезначимости}. $\models \varphi \Leftrightarrow$ таблица $T_\varphi = <\varnothing, \{\varphi\}>$ невыполнима.

\textbf{Закрытая семантическая таблица} $<\text{Г}, \Delta>$ --- семантическая таблица, у которой $\text{Г} \cap \Delta \neq \varnothing$

\textbf{Теорема о невыполнимости семантической таблицы}. Закрытая семантическая таблица невыполнима.

\textbf{Атомарная семантическая таблица} $<\text{Г}, \Delta>$ --- семантическая таблица, в которой множества Г и $\Delta$ состоят только из атомарных формул.

\textbf{Теорема о выполнимости семантической таблицы}. Незакрытая атомарная семантическая таблица выполнима.

\textbf{Логический вывод} --- доказательство общезначимости путем преобразования семантической таблицы $T_\varphi = <\varnothing, \{\varphi\}>$ к закрытым таблицам.

\textbf{Табличный вывод} --- такой вывод, в котором участвуют семантические таблицы.

\textbf{Корректный табличный вывод} --- такой табличный вывод, при котором правила преобразования таблиц (правила табличного вывода) сохраняют выполнимость семантических таблиц.

%%%%%%%%%%%%%%%%%%%%%%%%%%%Лекция 4%%%%%%%%%%%%%%%%%%%%%%%%%%%%%

\textbf{Подстановка} --- это всякое отображение $\theta : Var \rightarrow Term$, сопоставляющее каждой переменной некоторый терм.

\textbf{Область подстановки} $Dom_\theta$ --- множество $\{x : \theta(x) \neq x\}$

\textbf{Конечная подстановка} --- такая подстановка $\theta$, у которой $Dom_\theta$ --- конечное множество.

\textbf{Subst} --- множество конечных подстановок.

Переменная x называется \textbf{свободной для терма t} в формуле $\varphi(x)$, если любое свободное вхождение переменной x в формуле $\varphi(x)$ не лежит в области действия ни одного из квантора, связывающего переменную из множества $Var_t$.

Подстановка $\theta = \{ x_1/t_1 , \dots , x_n/t_n \}$ называется \textbf{правильной} для формулы $\varphi$, если для любой связки $x_i/t_i$ переменная $x_i$ свободна для терма $t_i$ в формуле $\varphi$.

Правила табличного вывода --- 12 правил (4.11-15)

\textbf{Табличный вывод} для таблицы $T_0$ --- это корневое дерево, вершинами которого служат семантические таблицы и при этом
\begin{enumerate}
 \item корнем дерева является таблица $T_0$ ;
 \item из вершины $T_i$ исходят дуги в вершины $T_j (T_k)~~\Leftrightarrow~~\frac{T_i}{T_j , (T_k)}$ --- правило табличного вывода.
 \item листьями дерева могут быть только закрытые и атомарные таблицы.
\end{enumerate}

\textbf{Успешный табличный вывод} (табличное опровержение) --- табличный вывод, в котором дерево вывода является конечным, а все его листья ---  закрытые таблицы.

Существование успешного вывода означает, что корневая семантическая таблица $T_0$ невыполнима.

\textbf{Лемма о корректности правил вывода}. Каково бы ни было правило табличного вывода $L\wedge, R\wedge, L\vee, R\vee, L\rightarrow, R\rightarrow, L\neg, R\neg, L\forall, R\forall, L\exists, R\exists~~~\frac{T_0}{T_1, (T_2)}$, таблица $T_0$ выполнима тогда и только тогда, когда выполнима таблица $T_1$ (или выполнима таблица $T_2$).

\textbf{Теорема корректности табличного вывода}. Если для семантической таблицы $T_0$ существует успешный табличный вывод, то таблица $T_0$ невыполнима.

%%%%%%%%%%%%%%%%%%%%%%%%%%%Лекция 5%%%%%%%%%%%%%%%%%%%%%%%%%%%%%

\textbf{Теорема полноты табличного вывода}. Если семантическая таблица $T_0$ невыполнима, то для $T_0$ существует успешный табличный вывод.

\textbf{Теорема Геделя (о полноте)}. Если формула $\varphi$ общезначима, то существует успешный табличный вывод для таблицы $T_\varphi = <\varnothing|\varphi>$.

\textbf{Теорема Левенгейма-Сколема}. Формула $\varphi$ выполнима $\Leftrightarrow \varphi$ имеет модель с конечной или счетно-бесконечной предметной областью.

\textbf{Теорема компактности Мальцева}. $\text{Г} \models \varphi ~~ \Leftrightarrow ~~ $ существует такое конечное подмножество $\text{Г}', \text{Г}' \subseteq \text{Г}$, что $\text{Г}' \models \varphi$


%%%%%%%%%%%%%%%%%%%%%%%%%%%Лекция 6%%%%%%%%%%%%%%%%%%%%%%%%%%%%%

\textbf{Эквиваленция} $\equiv$ --- выражение $\varphi \equiv \psi$ есть сокращенная запись $(\varphi \rightarrow \psi) \wedge (\psi \rightarrow \varphi)$

Формулы $\varphi$ и $\psi$ называются \textbf{равносильными}, если формула $\varphi \equiv \psi$ общезначима, то есть $\models \varphi \equiv \psi$.

Отношение равносильности --- это отношение эквивалентности.

Если $\varphi$ общезначима(выполнима) и $\varphi \equiv \psi$, то и $\psi$ общезначима (выполнима).

Запись $\varphi[\psi]$ означает, что формула $\varphi$ содержит подформулу $\psi$.

Запись $\varphi[\psi/\chi]$ обозначает формулу, которая образуется из формулы $\varphi$ заменой некоторых (не обязательно всех) вхождений подформулы $\psi$ на формулу $\chi$.

\textbf{Теорема о равносильной замене}. $\models \psi \equiv \chi ~ \Rightarrow ~ \models \varphi[\psi] \equiv \varphi[\psi / \chi]$

Замкнутая формула $\varphi$ называется \textbf{предваренной нормальной формой} (ПНФ), если $\phi = Q_1 x_1 Q_2 x_2 \dots Q_n x_n M(x_1, x_2 \dots x_n)$, где
\begin{itemize}
 \item $Q_1 x_1 Q_2 x_2 \dots Q_n x_n$ --- кванторная приставка, состоящая из кванторов $Q_1 Q_2 \dots Q_n$
 \item $M(x_1, x_2 \dots x_n)$ --- матрица --- базкванторная конъюнктивная нормальная форма (КНФ).
\end{itemize}

\textbf{Теорема о ПНФ}. Для любой замкнутой формулы $\varphi$ существует равносильная предваренная нормальная формула $\psi$.

\textbf{Сколемовская стандартная форма} (ССФ) --- предваренная нормальная форма, у которой в кванторной приставке отсутствуют кванторы существования: $\phi = \forall x_{i_1} \forall x_{i_2} \dots \forall x_{i_m} M(x_{i_1}, x_{i_2} \dots x_{i_m})$

\textbf{Теорема о ССФ}. Для любой замкнутой формулы $\varphi$ существует такая сколемовская стандартная форма $\psi$, что $\varphi$ выполнима $\Leftrightarrow \psi$ выполнима.

\textbf{Лемма об удалении кванторов существования}. Пусть $\phi = \forall x_1 \forall x_2 \dots \forall x_k \exists x_{k+1}  \varphi_0 (x_1, x_2 \dots x_{k+1})$ --- замкнутая формула, $k \geqslant 0$, и k-местный функциональный символ $f^{(k)}$ не содержится в формуле $\varphi$. Тогда формула $\varphi$ выполнима тогда и только тогда, когда выполнима формула $\psi = \forall x_1 \forall x_2 \dots \forall x_k \varphi_0(x_1, x_2 \dots x_k, f^{(k)}(x_1, x_2, \dots, x_k))$

\textbf{Сколемизация} --- процесс удаления кванторов $\exists$

\textbf{Сколемовский терм} --- терм $f^{(k)}(x_1, \dots, x_n)$, который подставляется вместо удаляемой переменной $x_{k+1}$, связанной квантором существования $\exists$.

Если $k=0$, то терм называется \textbf{сколемовской константой}.

\textbf{Теорема о невыполнимости сколемовской формулы}. Сколемовская стандартная форма $\varphi = \forall x_1 \forall x_2 \dots \forall x_m (D_1 \wedge D_2 \wedge \dots \wedge D_N)$ невыполнима тогда и только тогда, когда множество формул $S_{\varphi} = \{\forall x_1 \forall x_2 \dots \forall x_m D_1, \forall x_1 \forall x_2 \dots \forall x_m D_2, \dots, \forall x_1 \forall x_2 \dots \forall x_m D_N\}$ не имеет модели.

\textbf{Дизъюнкт} --- формула из множества $S_{\varphi}$. Дизъюнкт имеет вид $\forall x_1 \forall x_2 \dots \forall x_m (L_1 \vee L_2 \vee \dots \vee L_k)$, где $L_i$ --- литера, которая является либо атомом, либо отрицанием атома.

\textbf{Пустой дизъюнкт} --- дизъюнкт, не имеющий литер и тождественно ложный.

\textbf{Невыполнимая (противоречивая) система дизъюнктов} --- система дизъюнктов, не имеющая модели.

%%%%%%%%%%%%%%%%%%%%%%%%%%%Лекция 7%%%%%%%%%%%%%%%%%%%%%%%%%%%%%

\textbf{Теорема о противоречивости системы дизъюнктов}. Система дизъюнктов $S = \{D_1, D_2, \dots, D_n\}$ протеворечива тогда и только тогда, когда для каждой интерпретации I в системе S найдется такой дизъюнкт $D_i = \forall x_1 \forall x_2 \dots \forall x_m (L_{1i} \vee L_{2i} \vee \dots \vee L_{k_ii})$, а в предметной области такой набор $d_1, \dots, d_n$, для которых имеют место: $I \nvDash L_{1i}[d_1, \dots, d_n] ~~ \dots ~~ I \nvDash L_{k_ii}[d_1, \dots, d_n]$


\textbf{Эрбрановские интерпретации} (H-интерпретации) --- это специальная разновидность интерпретаций, в основе которых лежат свободные алгебры.

\textbf{Эрбрановский универсум} (H-универсум) --- предметная область эрбрановских интерпретаций.

\textbf{Эрбрановским универсумом} $H_\sigma$ для сигнатуры $\sigma = <Const, Func, Pred>$ называется множество $H_\sigma = \bigcup\limits_{i=0}^{\infty}H_i$, где
\begin{itemize}
 \item $i=0 ~~ H_0 =
\begin{cases}
Const	&	\text{если } Const \neq \varnothing \\
\{c\}	&	\text{если } Const = \varnothing (\text{эрбрановская константа})\\
\end{cases}
$
 \item $i \rightarrow i+1 ~~ H_{i+1} = H_i \cup \{f^{(k)}(t_1, t_2, \dots, t_k) : t^{(k)} \in Func ~,~ t_1,t_2,\dots,t_k \in H_i\}$
\end{itemize}

\textbf{Основной терм} --- терм эрбрановского универсума. Не содержат переменных.

\textbf{Эрбрановская интерпретация} $I_H = <H_\sigma, \overline{Const}_H, \overline{Func}_H, \overline{Pred}>$ сигнатуры $\sigma = <Const, Func, Pred>$ состоит из
\begin{itemize}
 \item стандартной предметной области --- эрбрановского универсума $H_\sigma$
 \item стандартной оценки констант $\overline{Const}_H(c) = c$, т.е. значением каждого константного символа является его отображение
 \item стандартной оценки функциональных символов $\overline{Func}_H(f^{(n)}) = \overline{f} : \overline{f}(t_1, \dots, t_n) = f^{(n)}(t_1, \dots, t_n)$, то есть каждый функциональный символ $f$ играет роль конструктора термов эрбрановского универсума
 \item произвольной оценки предикатных символов
\end{itemize}

\textbf{Теорема о H-интерпретациях}. Система дизъюнктов S выполнима тогда и только тогда, когда имеет эрбрановскую модель, т.е. выполнима хотя бы в одной H-интерпретации.

\textbf{Основной атом} --- формула $P^{(m)}(t_1, \dots, t_m)$, где $P^{(m)} \in Pred~,~t_1,\dots,t_m \in H_\sigma$ (набор основных термов)

\textbf{Эрбрановский базис} ($B_H$) --- набор всех основных атомов.

Всякая H-интерпретация I задается подмножеством $B^I$ истинных основных атомов: $B^I = \{P^{(m)}(t_1,\dots,t_m) : I \models P^{(m)}(t_1,\dots,t_m), t_1,\dots,t_m \in H\}$

\textbf{Основной пример дизъюнкта} $D = \forall x_1 \forall x_2 \dots \forall x_m (L_1 \vee L_2 \vee \dots \vee L_k)$ --- дизъюнкт $D' = (L_1 \vee L_2 \vee \dots \vee L_k)\{x_1/t_1,\dots,x_m/t_m\}$, полученный из D путем подстановки основных термов $t_1,\dots,t_m \in H_\sigma$ эрбрановского универсума.

\textbf{Теорема о противоречивости системы дизъюнктов (через эрбрановские интерпретации)}. Система дизъюнктов $S = \{D_1, \dots, D_N\}$ противоречива\\
$\Leftrightarrow$\\
для каждой H-интерпретации в S найдется такой дизъюнкт $D = (L_{1i} \vee L_{2i} \vee \dots \vee L_{k_ii})$ и такой набор основных термов $t_1, \dots, t_m$, для которых имеет место $I \nvDash (L_{1i} \vee L_{2i} \vee \dots \vee L_{k_ii})[t_1,\dots,t_m]$\\
$\Leftrightarrow$\\
для каждой H-интерпретации существует основной пример $D' = D_i\{x_1/t_1,\dots,x_m/t_m\}$ дизъюнкта $D \in S$, для которого $I \nvDash D'$

\textbf{Теорема Эрбрана}. Система дизъюнктов $S = \{D_1, \dots, D_N\}$ противоречива тогда и только тогда, когда существует конечное противоречивое множество $G'$ основных примеров дизъюнктов S.

\textbf{Следствие теоремы Эрбрана}. Система дизъюнктов S противоречива тогда и только тогда, когда S невыполнима ни в одной эрбрановской интерпретации.

\textbf{Композиция подстановок} $\mu = \theta\eta~,~ \theta,\eta \in Subst$, которая определяется как $\mu(x) = (x\theta)\eta$ для любой $x \in Var$.

Утверждение. Пусть $\theta = \{x_1/t_1,\dots,x_n/t_n\}~,~\eta = \{y_1/s_1,\dots,y_m/s_m\}$, тогда подстановка $\mu = \{x_1/t_1\eta,\dots,x_n/t_n\eta\} \cup \{y_i/s_i : y_i\notin\{x_1,\dots,x_n\}\} - \{x_j/t_j\eta : x_j = t_j\eta\}$.

\textbf{Унификатор} --- подстановка $\theta$ логических выражений $E_1,E_2$ такая, что $E_1\theta = E_2\theta$.

\textbf{Наиболее общий унификатор} (НОУ) --- унификатор выражений $E_1,E_2$ такой, что для любого унификатора $\eta$ для этих же выражений существует подстановка $p$ такая, что $\eta = \theta p$ является композицией $\theta$ и p.

\textbf{Задача унификации} --- для двух выражений $E_1,E_2$ выяснить, являются ли они унифицируемыми и, в случае унифицируемости, вычислить наиболее общий унификатор.

%%%%%%%%%%%%%%%%%%%%%%%%%%%Лекция 8%%%%%%%%%%%%%%%%%%%%%%%%%%%%%

\textbf{Лемма о связке}. Пусть $x \in Var, t \in Term$, тогда
\begin{enumerate}
 \item если $x \notin Var_t$, то $\{x/t\} \in \text{НОУ}(x,t)$
 \item если $x \in Var_t~,~x \neq t$, то $\text{НОУ}(x,t) = \varnothing$
\end{enumerate}

\textbf{Приведенная система} --- система уравнений
$
E =
\begin{cases}
x_1 = s_1 \\
x_2 = s_2 \\
\dots \\
x_n = s_n \\
\end{cases}
$
и при этом
\begin{itemize}
 \item $x_1, \dots, x_n \subseteq Var$
 \item все переменные $x_1, \dots, x_n$ попарно различны
 \item $\{x_1, \dots, x_n\} \cap \bigcup\limits_{i=1}^n Var_{s_i} = \varnothing$
\end{itemize}

\textbf{Лемма о приведенной системе}. Если система E является приведенной, то унификатор $\{x_1/s_1,\dots,x_n/s_n\}$ является ее наиболее общим унификатором.

\textbf{Равносильные системы} $E_1,E_2$ --- такие системы, что $\text{НОУ}(E_1) = \text{НОУ}(E_2)$

\textbf{Алгоритм унификации (алгоритм Мартелли-Монтанари)} --- недетерменированный алгоритм из 6 правил, которые можно применять в любом порядке до тех пор, пока ни одно из правил нельзя применить или не применялось одно из правил, устанавливающих невозможность унификации.

\textbf{Теорема об унификации}. Какова бы ни была система уравнений E
\begin{enumerate}
 \item Алгоритм Мартелли-Монтари всегда завершает свою работу
 \item Если система Е унифицируема, то в результате работы алгоритма будет построена приведенная система, равносильная Е.
 \item Если система Е неунифицируема, то в результате работы алгоритма применено правило, устанавливающее невозможность унификации.
\end{enumerate}

%%%%%%%%%%%%%%%%%%%%%%%%%%%Лекция 9%%%%%%%%%%%%%%%%%%%%%%%%%%%%%

\textbf{Переименование} --- подстановка $\theta : Var \rightarrow Var$ такая, что $\theta$ является биективным отображением.

\textbf{Пример выражения} Е --- выражение $E\theta$

\textbf{Основной пример} --- пример такой, что $Var_{E\theta} = \varnothing$.

\textbf{Вариант выражения} --- выражение $E\theta$, если $\theta$ является переименованием.

\textbf{Пустая (тождественная) подстановка} --- переименование.

\textbf{Правило резолюции}. $\frac{D_1, D_2}{(D_1' \vee D_2')\theta}$, где:
\begin{itemize}
 \item $D_1 = D_1' \vee L_1~,~D_2 = D_2' \vee \neg L_2$ --- два дизъюнкта
 \item $\theta = \text{НОУ}(L_1,L_2)$
\end{itemize}

\textbf{Резольвента дизъюнктов} $D_1,D_2$ --- дизъюнкт $D_0 = (D_1' \vee D_2')\theta$

\textbf{Контрактная пара} --- пара литер $L_1$ и $\neg L_2$

\textbf{Правило склейки}. $\frac{D_1}{(D'_1 \vee L_1)\theta}$, где:
\begin{itemize}
 \item $D_1 = D_1' \vee L_1 \vee L_2$ --- дизъюнкт
 \item $\theta = \text{НОУ}(L_1,L_2)$
\end{itemize}

\textbf{Склейка дизъюнкта} $D_1$ --- дизъюнкт $D_0 = (D_1' \vee L_1)\theta$

\textbf{Склеиваемая пара} --- пара литер $L_1$ и $L_2$

\textbf{Резолютивный вывод системы дизъюнктов} $S = \{D_1, \dots, D_N\}$ --- конечная последовательность дизъюнктов $D'_1, D'_2, \dots, D'_n$ (резолютивно выводимые дизъюнкты) такая, что для любого $i, 1 \leqslant i \leqslant n$ выполняется одно из трех условий:
\begin{enumerate}
 \item либо $D'_i$ --- вариант некоторого из дизъюнктов из S
 \item либо $D'_i$ --- резольвента дизъюнктов $D'_k, D'_j~,~ k,j < i$
 \item либо $D'_i$ --- склейка дизъюнктов $D'_j~,~j<i$
\end{enumerate}

\textbf{Успешный резолютивный вывод} (резолютивное опровержение) --- резолютивный вывод, который оканчивается пустым дизъюнктом $\Square$.

\textbf{Теорема корректности резолютивного вывода}. Если из системы дизъюнктов S резолютивно выводим пустой дизъюнкт $\Square$, то S --- противоречивая система дизъюнктов.

%%%%%%%%%%%%%%%%%%%%%%%%%%%Лекция 10%%%%%%%%%%%%%%%%%%%%%%%%%%%%%

\textbf{Теорема о полноте резолютивного вывода}. Если S --- противоречивая система дизъюнктов, то из S резолютивно выводим пустой дизъюнкт $\Square$.

Метод резолюций: корректен, полон, алгоритмизируем.

%%%%%%%%%%%%%%%%%%%%%%%%%%%Лекция 11%%%%%%%%%%%%%%%%%%%%%%%%%%%%%

\textbf{Полная стратегия резолютивного вывода} --- такая стратегия, которая позволяет вывести пустой дизъюнкт $\Square$ из любой противоречивой системы дизъюнктов.

\textbf{Семантическая резолюция} (11.5)

\textbf{Теорема полноты I-резолюции}. Если система дизъюнктов S противоречива, то для любой интерпретации I существует успешный I-резолютивный вывод пустого дизъюнктов $\Square$ из S.

\textbf{Линейная резолюция} (11.8)

\textbf{Теорема полноты линейного резолютивного вывода}. Если система дизъюнктов S противоречива, а система дизъюнктов $S \backslash \{D_0\}$ непротиворечива, то существует успешный линейный резолютивный вывод пустого дизъюнкта $\Square$ из S.

% $\not\models \emptyset$
% $\cal{G'}$
% готичские буквы
\end{document}